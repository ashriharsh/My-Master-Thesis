\chapter{Implications} \label{chap:chapter_7}
This chapter mainly focuses on what theoretical and practical implications can be drawn from the this work.

\section{Theoretical Implications} \label{sect:thefirst}

Additionally, there’s also a limitation in terms of the difficulty to generalize hyper-personalization for all types of products and customer profiles, which essentially means that segmentation would still be required as a first step before implementing hyper-personalized and hyper- contextualized targeting.
It therefore seems unrealistic to get rid of segmentation entirely, even though there’s still a need to adapt it to the new context: for example, usage-based segmentation will enable companies to identify customers who are eligible for hyper-personalization. Hyper-personalization is only relevant if a precise scope has already been defined and is completely understood by the marketing and communication teams: segmentation is a proven technique to establish these pre-requisites, and is essentially complementary to hyper-personalization rather than a conflicting
\autocite[3]{CapgeminiconsultingESSECBusinessSchool2016}


\section{Practical Implications}
What does it mean for RTI sports 

Context-Aware Recommender Systems 
How Contextual Factors Change over Time
Depending on whether contextual factors change over time or not, we have the following two cate- gories: static and dynamic. Static: The relevant contextual factors and their structure remains the same (stable) over time. For example, in case of recommending a purchase of a certain product, such as a shirt, we can include the contextual factors of Time, PurchasingPurpose, ShoppingCompanion and only them during the entire lifespan of the purchasing recommendation application. Dynamic: This is the case when the contextual factors change in some way. For example, the rec- ommender system (or the system designer) may realize over time that the ShoppingCompanion factor is no longer relevant for purchasing recom- mendations and may decide to drop it. Further- more, the structure of some of the contextual fac- tors can change over time (for example, new categories can be added to the PurchasingPurpose contextual factor over time).



