\chapter{Introduction} \label{chap:intro}
\onehalfspacing
\section{Problem Definition and Background} \label{sect:thefirst} 
The extremely competitive business environments have forced organizations to undertake strategies that would help them stay at the forefront of the competition. Companies are continuously exploring new marketing strategies to attract and retain their consumers. In spite of the different strategies employed by companies, it is still a challenging task to maintain a long-lasting relationship with consumers. A research conducted by \textcite[8-9]{Aziz2009} implies that the reasons for the decreasing customer loyalty are ignoring factors such as Service Process and Service Quality. Moreover, just satisfying the needs of the consumer is not enough to retain them. For example, an empirical study by \textcite[135]{Reichheld2000} found a customer defection rate of 60-80\% where the consumers were either satisfied or very satisfied. Thus, just satisfying the needs of the consumer is not enough and it is crucial for the success of organizations to focus on aspects that directly relate to gaining the consumer loyalty.\\ \par
\noindent
Due to the increasing domestic and international competition and high consumer defection rate many organizations are focusing on creating a delightful experience for their consumers than just satisfying them \autocite[386]{Bartl2013}. Literature suggests that a delightful experience for consumer is directly related to long-term customer relationship, loyalty and high purchase intentions as compared to customer satisfaction \autocite[386]{Bartl2013}. Although the constructs of consumer satisfaction and consumer delight are related but they have different effect on consumer behavior intention\autocite[392]{Bartl2013}. Consumer satisfaction occurs when the services received by the consumer exceeds their expectations whereas consumer delight is experiencing a positive surprise which surpass the expectations \autocite[129]{Matzler1996}. Consequently, providing an exceptional service quality that exceeds the customers expectations which not only satisfies them but also creates an delight is referred as Service Excellence \autocite[475]{Horwitz1996}. \\ \par
\noindent
There is a growing body of literature that recognizes the importance of Service Excellence in the area of Marketing. Many researchers have extensively studied it and have helped in improving the understanding of Service Excellence. \textcite[331]{Oliver1997} identifies Service Excellence as an unexpected experience which is a precursor to a delightful experience conceivably leading to higher customer retention.\textcite[130]{Johnston2004} in his work highlights the inadequacies in the definition of the term Service Excellence which is often translated to exceedingly satisfying service. In his opinion Service Excellence is nothing but a simple and hassle-free way of doing business. He further identifies four key elements of Service Excellence namely: delivering the promise, providing a personal touch, going the extra mile and resolving the problem. Considering this ideology and specifically focusing on the aspects of providing a personal touch, this work explores how Artificial Intelligence can help in delivering a positive and unexpected experience which ultimately lead to achieving Service Excellence. \\ \par

\noindent
Through the widespread existence of the internet and mobile devices, it has become much easier to connect and to reach a wider and diverse set of audience. Taking advantage of this technological evolution, many organizations are actively using electronic channels as a mean to reach to their customers. Although the gradual shift from the traditional Brick and Mortar Retail stores to Electronic stores has revolutionized the marketplace, this new style has its own set of pros and cons \autocite[p. 2362 ff.]{Niranjanamurthy2013}. To make the  electronic channel more appealing to the customers, companies must try to address the drawbacks of e-commerce such as a lack of human touch \autocite[1153]{Qiu2006}. It is of utmost importance to understand the needs of the consumer and identify the factors which lead to higher consumer satisfaction. These factors can be further exploited to develop longstanding relations with customers \autocite[284]{Thaichon2014}.  \\ \par

\noindent
E-commerce platforms provide an opportunity to aid customers with a wide variety of options related to companies offering and services. However, with this enormous amount of information at disposal, it usually leads to information overload \autocite[158]{SchaferJoshephKonstan1999}. Given this load of information and limited attention span of consumers, the challenge of delivering the most appropriate content at the right time arises. Thus, there is a need to filter unnecessary information and retain information which is related to the interest of the consumer. Removing or filtering such information is accomplished using intelligent systems based on machine learning algorithms which predict items/information related to the interest areas of the customer \autocite[103]{Konstan2012}. Such systems are identified as Recommender systems. \\ \par
\noindent
Recommender systems are used extensively by large organizations such as Amazon, eBay, Netflix, Spotify, YouTube, etc. to provide personalized information/content to each of its users. Such systems try to identify the interest of the user based on the past buying pattern, demographics of the customers, the popularity of the product to predict the buying action \autocite[158]{SchaferJoshephKonstan1999}. Thanks to the recent advancement in the field of Artificial Intelligence and Machine Learning, it is technically possible to industrialize the process of personalizing the content for each user without being explicitly programmed to do so \autocite[104]{Konstan2012}. Furthermore, the rising popularity of the electronic channels among the population makes it easy to gather comprehensive information such as ‘customers’ browsing the history’, and ‘buying patterns’. Such detailed data is the vital component to develop an efficient recommender system which tries to model user behavior \autocite[116]{Schafer2001}.  \\ \par
\noindent
The traditional marketing strategies such as RFM (Recency, Frequency, Monetary) segmentation aims at dividing the consumers into subgroups based on certain characteristics such as age, gender, occupation, geographic location \autocite[]{CapgeminiconsultingESSECBusinessSchool2016}. Although such strategies have been successful in the past, they fail to capture the dynamic behavior of the customers, where the preferences of customers are heavily influenced by external factors. Thus, making the traditional marketing strategies outdated, and calls for a new approach in marketing. One such approach is Hyper-Personalization \autocite[2]{Sodhi}. It essentially aggregates contextual information such as time of the day, demographic information along with the purchase history to model the preferences of the user, thus personalizing the offerings for an individual consumer (Capgemini Consulting and ESSEC Business School, 2016, p. 5). Traditional Recommender Systems do not consider contextual information in recommending items/information \autocite[4-5]{Adomavicius2005} (Adomavicius et al. 2005, pp. 4-5) . In applications such as a recommending vacation package, personalized content on Web Pages or even recommending a movie or music, it may not be sufficient enough to just consider the previous observed patterns. For e.g. in a news recommender application where a user may prefer to read world news in the morning, financial reports in the afternoon and movie review on weekends. Such applications must not only identify the interests of the users but also adjust the recommendations according to the situational context i.e. time, location, behavior, mood \autocite[2]{Adomavicius2005} (Adomavicius, Sankaranarayanan, Sen and Tuzhilin, 2005, p. 2). A comprehensive study on the importance of contextual information in recommendations conducted by Adomavicius et al. (2005, pp. 2-10) highlights the fact that the consumers decision making process is dependent on a certain context. Thus, accurately predicting the preferences of the customer is directly related to the range of contextual information included in the recommendation process. A Context-Aware Recommender System extends the classic Recommender System by integrating contextual information. This understanding can be associated with the concept of Hyper-Personalization of services. \\ \par
\noindent
The ongoing development in the field of Artificial Intelligence (AI) has found a useful application in various industries ranging from health to e-commerce. The 9-fold increase in the number of research articles related to AI from 1996-2017 indicates the high interest of AI among the research community \autocite[9]{Shoham2018}. The field of Marketing and Sales is no stranger to the use of AI. The ability to collect data, analyze, operationalize and, learn from it using AI tools has helped companies to define and improve their marketing strategies. Considering the basic principle of marketing i.e. focusing on the customer and their needs, the capability of such tools in personalizing content for each user paves a path for the companies to provide a personalized user experience. When relevant content is presented to the user, they are more likely to convert and become recurring customers and have brand loyalty \autocite{content} (Content Marketing Institute, 2016) . Thus, the innovative and process-oriented use of such tools within the framework of basic sales strategy decisions, to sustainably increasing sales results is nothing but Digital Sales Excellence \autocite[5]{Binckebanck2016}.
There is abundant literature available describing in theory what Service Excellence is, but very few describe the practical implementation of important constructs that are crucial in leading to Service Excellence\autocite[131]{Johnston2004}. Bearing this knowledge in mind, this study focuses on understanding the role of hyper-personalized systems in adding a personal touch which is a precursor in achieving Digital Sales Excellence. 




\section{Research Objective and Research Question}

The increasing competition in the online retail space has led organizations to not only focus on the business aspects but also on providing excellent service to customers and improving the overall experience. A study conducted by \textcite[117]{Schafer2001} describes the advantages of enabling a customer-centric process and its impact on building long-lasting relations, increasing cross-selling behavior and converting browsers into buyers. The popularity of electronic-commerce among the consumers has paved path for new marketing strategies that could not be devised and implemented in traditional Brick and Mortar stores \autocite[806]{Hosanagar2013} (Hosanagar et al. 2013, p. 806). It is now feasible to capture the ever-changing preferences of the customer using technologies such as recommender systems. Such systems enable this radical shift from a traditional mass segmentation marketing technique towards a micro-segmentation one, focusing on an individual level \autocite[5]{CapgeminiconsultingESSECBusinessSchool2016}. \\ \par 
\noindent
RTI Sports GmbH is transitioning from a segment-based marketing technique to a more user-centric process. RTI (Radsport, Triathlon, Innovation) Sports GmbH, founded by Mr. Francis Arnold, is a major player in manufacturing high-quality bicycle accessories. The company was founded in the year 1990 and has grown noticeably, housing eight strong premium brands under one roof. RTI Sports consistently emphasizes on manufacturing products of high standards and quality within a segment to meet the demands of the customer.  \\ \par
\noindent
The business model of RTI Sports is retailing and manufacturing. They employ both B2B (Business2Business) and B2C (Business2Customer) business models. However, the majority of the sales marketing activities are focused on the B2B business model. The business partners in the B2B setting are a large network of local and global dealers. A strong sales team at RTI sports manage the sales and marketing functions for all dealers. However, the ever-increasing range of products makes it difficult to curate the best product offering for each dealer. Moreover, with the planned B2C business model via the e-commerce platform, it will be crucial to include a product Recommender System and have a competitive edge over the others. With an intention to further strengthen the relationship between the dealers and create a personalized offering to the dealers, RTI sports has decided to adopt a new data-driven strategy. The application of a Context Aware Recommender System would help in better understanding the needs of their dealers and recommend appropriate products. Implementing such system will reduce the information overload on the sales team where they have to suggest a suitable product from a large catalog of items in a limited time span. Thus, the use of such AI system within the framework of basic sales strategy decision would lead to Digital Sales Excellence.  \\ \par
\noindent
The objective of this work is to review the relevant scientific literature related to Artificial Intelligence, Hyper-personalization, Digital Sales Excellence in-order to create a structured summary of research work previously done and analyze the current state of art work related to Recommender Systems which can be utilized to develop a Context Aware Recommender System.  \\ \par
\noindent
The main theme of this study can be formulated as:  \textbf{\emph{Study the effectiveness of Context-Aware Recommender Systems (CARS) in the context of Hyper-Personalization of services and Digital Sales Excellence.}} \\ \par
\noindent
Therefore, the main research questions defined considering the main theme of this work are as follows: \\ 

\textbf{\emph{RQ1: How can a Context-Aware Recommender System be used to implement Hyper-Personalization in a Sales context?}} \\

\textbf{\emph{RQ2: Which effects does such a system have on the Digital Sales Excellence from a Management’s perspective?}} \\ 

\textbf{\emph{RQ3: How does a Context-Aware Recommender System influence the perception of hyper-personalized offerings from the dealers’ perspective? Which implications can be derived from this evaluation compared to “traditional recommender systems”?}}


\section{Methodology} 
The Methodological approach adopted in this study can be broadly classified into 3 phases namely: Literature Review, Implementation and Experiments, and Evaluation.
 \subsection{Literature Review}
 In any research work, the most important step is to review the literature related to the topic which sets the theoretical background for the work.  Since the main focus of this research work is to investigate and study the effects of Hyper-Personalization in achieving Digital Sales Excellence, it is imperative to review the literature related to Service Excellence, Hyper-Personalization and, Artificial Intelligence which helps in realizing the concept. It is envisioned to review the literature that not only identifies the major research work accomplished, but also the research gap.  A systematic literature review helps in collecting and producing previous research done. Therefore the analysis of the literature will be based on the approach suggested by Snyder (2019, pp. 336-338). According to the framework the entire process of analyzing the literature is categorized into four phases namely: designing the review, conducting the review, analysis and, writing the review. Each phase consists of a set of important questions which helps in identifying the main contributions and filter the irrelevant ones. 
Initially the research articles will be sourced from online digital libraries such as Google Scholar, ResearchGate, JSTOR, arXiv, Association for Computing Machinery (ACM), CiteSeerX.  Relevant journals (Journal of Services Marketing Managing Service Quality, Journal of Marketing, Journal of Retailing, International Journal of Service Industry Management, Data Mining and Knowledge Discovery, Artificial Intelligence Review, ACM Transactions on Interactive Intelligent Systems, ACM RecSys) will be considered which have a focus on Service Excellence, Artificial Intelligence and Recommender Systems. To identify the relevant literature, search terms such as Service Excellence, Customer Satisfaction, Hyper-Personalization, Recommender System, Context-Aware Recommender System, which are directly related to the theme of the study and research questions, will help to gain access to appropriate articles, books, and reports. On further analysis, the shortlisted articles will be validated for its contributions and relevance to the topic. Bearing the research questions in mind, information such as major findings and effects, topic, type of study etc. will be abstracted to answer the research questions. Finally, the information gathered will be summarized which can be extended to answer the research questions and form a solid base for the implementation step.
\subsection{Implementation and Experiments}
Although the Hyper-Personalization concept is relatively new, the underlying paradigm relies on utilizing technologies such as Machine Learning and Big Data. More specifically, Recommender Systems have proved to be very efficient in the area of personalizing information based on the past purchase / interaction history (Konstan and Riedl, 2012, p. 102). Such systems utilize algorithms such as Matrix Factorization, K-Nearest Neighbors, Bayesian Probabilistic approach. Additionally, the type of data available further classifies the recommender systems into 2 main categories namely implicit and explicit. Implicit recommender systems model users’ preferences using information such as likes / dislikes, which are mostly binary in nature (Bobadilla et al. 2013, p. 110). Similarly, explicit recommender systems utilize data such as ratings and reviews given by customers for a product (Hu, Koren and Volinsky, 2008, p. 263) .
Considering this fundamental difference between the type of data, and the unavailability of explicit data, modeling Recommender systems that utilize implicit data is selected to be implemented at RTI Sports. The implicit data is derived from the past purchase across the 3-year time frame.  This work forms the base model which will be used later for the evaluation purpose. 
To model a Context-Aware Recommender System, first it is necessary to obtain the contextual information. Temporal information associated with the purchase history is a widely-used format of contextual information. Furthermore, to provide a real-time personalized recommendation, contextual information such as time is used along with the past purchase history to predict users buying behavior, thus addressing the Hyper-Personalization concept. Accompanying this step, evaluating the performance of the algorithm under different algorithm parameters completes the implementation phase. Finally, the performance of Context-Aware Recommender Systems and Traditional Recommender Systems is compared to measure the effect of contextual information in a recommendation process.

\subsection{Evaluation}
This phase covers the details of the experimental design setup used to evaluate and measure the effectiveness of the proposed system. Since recommendations is subjective in nature, it is proposed to conduct qualitative interviews with the sales team at RTI sports to get their opinion on the recommendations generated. The interview questionnaire will be drafted based on the defined interview guideline which not only helps in evaluating the effect of the implemented recommender system but also covers the different aspects in the context of reaching Digital Sales Excellence. Moreover, the evaluation phase brings together the results of the literature analysis and the qualitative information derived from the questionnaires.
Additionally, to verify the capability of Context Aware Recommender Systems in identifying the preferences of the consumer, the recommendations will be compared with that from a traditional Recommender System. Metrics such as Root Mean Squared Error, Normalized Discounted Cumulative Gain at K and Mean Average Precision at K help in measuring the effectiveness of such systems.



\section{Course of the Investigation}
The research work is divided into 7 main chapters. Each chapter describes its value and purpose and creates a basic background for the upcoming chapters. Chapter 1 gives a general introduction to the topic of this master’s thesis. It also establishes the statement of the problem, its relevance and discusses the research objective and questions covered in this work. Chapter 2 forms the basis for the literature review and introduces the reader to the key terminologies used in this work and also the research literature related to the topic from the Marketing perspective, the Artificial Intelligence perspective and the usage of Artificial Intelligence approaches in Marketing. This chapter also identifies the relevant and most significant research work. Chapter 3 explains the research methods and the system architecture implemented to study the topic. It also further enlarges on the experiments conducted with different settings and lastly discusses the results of experiments.
Chapter 4 defines various evaluation techniques used to measure the effectiveness of the approach defined in Chapter Three and also presents the findings of the experiments and the interviews. Chapter 5 debates the result obtained in Chapter Four considering the research questions that were lectured in Chapter 1 and draws conclusions about the study using the Quantitative analysis and the findings from the experiments. Chapter 6 summarizes the theoretical and practical implications associated with the study and future scope of improvement. Finally Chapter 7 provides the final summary on the study and also highlights limitations of the work. 
